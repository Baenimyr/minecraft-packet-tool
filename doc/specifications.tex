\documentclass{article}
\usepackage{inputenc}
\usepackage[T1]{fontenc}
\usepackage[margin=25mm]{geometry}
\usepackage[french]{babel}
\usepackage{hyperref}
\usepackage{algorithm2e}
\usepackage{amssymb}


\usepackage{enumitem}
\usepackage{dirtree}

\setitemize{noitemsep,topsep=0pt,parsep=0pt,partopsep=0pt}

\title{Document de spécification pour le projet ForgeMods}

\begin{document}
\maketitle
\tableofcontents

\section{Description du projet}
MinecraftForge est le système de mod autour de minecraft le plus utilisé au monde.
La communauté est très grande et le développement produit une très grande quantité de mods.

Plus particulièrement, au moment de rejoindre un serveur moddé, certains peuvent manquer et les dépendances des mods entre eux n'est pas évidente à résoudre.
Il faut attendre le lancement du jeu pour ce rendre compte qu'il en manque un, et ensuite la dépendance de la dépendance.

L'objectif est un programme en ligne de commande capable de télécharger n'importe quel mod (suffisamment connu) dans le dossier de son choix en quelques secondes.
Le système Linux, propose des gestionnaire de paquets capable également de résoudre les dépendances au moment de l'installation, ainsi rien ne manque.


\section{Fonctionnalités}
Les fonctionnalités suivantes sont présentées dans l'ordre de complexité et l'ordre de réalisation.

\subsection{Analyse de l'installation}
Il s'agit ici de lire les fichiers \texttt{jar} présent dans le répertoire d'installation et d'en extraire des données (voir \ref{subsection:depot_installation}).
Ensuite le système peut vérifier l'arbre des dépendances et informer l'utilisateur de l'absence d'un mod particulier.

Une fois les informations extraites, le programme peut afficher les informations, afficher l'arbre de dépendance, faire des estimations de ram nécessaire au lancement du jeu.
MinecraftForge est déjà capable de vérifier les mises à jour et ce programme pourrait également le faire.


\subsection{Arbre de dépendance}
L'arbre de dépendance permet de calculer les intervalles de version satisfaisants plusieurs contraintes.
Si plusieurs mods déclarent des dépendances sur un même mod mais pour des intervals différents, le compromis correspond à l'intersection des intervalles.



\subsection{Modification}
\subsubsection{Installation}
Lorsque un utilisateur le demande, il faut récupérer l'archive jar d'un mod et la placer dans le dossier \textit{mods}.
Cependant il faut maintenir le bon fonctionnement de l'installation, notamment la gestion des dépendances.
En considérant la liste des mods installés manuellement avec le nouveaux mods en cours d'installation, il faut calculer les dépendances nécessaires.
Si aucune version connue n'est capable de satisfaire les dépendances connues, c'est une erreur.

Les versions installées manuellement peuvent ne plus être compatibles avec les nouveaux mods, mais puisque ils ont été choisis par l'utilisateur, il ne seront pas modifiés autrement que manuellement.
Seule les dépendances non satisfaites par l'installation actuelle seront installées, il ne s'agit pas ici de mettre à jour si ce n'est pas nécessaire.


\subsubsection{Mise à jour}
La mise à jour automatique consiste à détecter la disponibilité d'une version plus récente pour les mods installés.
Les dépendances sont calculées à partir des nouvelles versions pour les installations manuelle.
Dans le cas d'une mise à jour générale et contrairement à l'installation standard, même les installations automatiques seront mises à jour dès que possible.

\subsubsection{Suppression}
Pour éviter les suppressions qui rendraient l'installation invalide, la suppression consiste à marquer un mod comme non installé manuellement.
Ainsi, s'il s'agit d'une dépendance pour un autre mod toujours présent, il ne sera pas supprimé.
Par contre si aucun des mods consistuant la \textit{racine} de l'installation n'a besoin de lui, le fichier est supprimé.

La suppression des dépendances devenues inutiles peut se faire avec la commande \texttt{autoremove} (non implémentée).


\subsection{Téléchargement des fichiers}
Une fois un mod demandé ou requis par les autres, le programme doit pouvoir trouver le fichier correspondant sur internet.
Pour un identifiant de mod et une version, il faut maintenant fournir l'url de téléchargement (par exemple directement les sites hébergeurs).
Les versions disponibles peuvent être limitées aux versions stables.

\subsection{Analyse des demandes pour serveur sous MinecraftForge}
Les serveurs MinecraftForge transmettent la liste des mods requis lors des opérations de salutations (nom, nombre de joueurs, \dots).
Avec une adresse de serveur fournies, le programme peut vérifier la disponibilité des mods et proposer d'en télécharger.


\section{Dépôts}
\label{section:depot}
Malheureusement trop souvent, les informations des fichiers \textit{mcmod.info} sont mal renseignées et les vrais données sont fournies à \textit{@Mod} dans le code.
Cette source d'information est inaccessible.
S'ajoute à cela les informations comme le lien de téléchargement qui ne sont pas fournies dans le fichier jar.
L'ensemble des informations vérifiées devront donc être disponibles en ligne, et localement en cache.

\subsection{Dépôt internet}
Une base de données en ligne pose des problèmes de complexité de lecture et de mise en place.
En s'inspirant des paquets Linux toujours, le téléchargement de fichiers individuels dans une arborescence structurée est plus adéquate.
Un premier fichier texte compressé, recense les mods disponibles dans un dépôt (voir \texttt{Packages.xz}).
Pour un mod et une version connue, correspond un emplacement dans l'arbre des arborescence.
Un système basé sur les dépôt permet aux groupes de développeurs de gérer leur propre dépôt.
L'utilisation d'un dossier intermédiaire avec la première lettre du modid permet une gestion mémoire physique plus souple sur les grands dépôts.

\subsection{Dépôt local}
Sans connexion internet, le programme doit pouvoir fournir les informations détaillées glanées sur internet et concernant les mods couremment installés.
Il est inutile de s'encombrer des informations non nécessaires.
Les informations locales peuvent être mises à jour sur demande ou automatiquement de temps en temps.

La synchronisation se fait en lisant les fichiers du dépôt distants et en ajoutant au dépot actuel les nouvelles informations, ainsi en cas de conflit entre différents dépôt, le premier de la liste est utilisé.
De préférence, les fichiers du dépôt doivent être rassemblé dans un fichier tar pour accélérer le téléchargement: une seule connexion seulement doit être établie au lieu d'une connexion par fichier.
Pour un dépôt correct, le dépôt local doit être effacé pour être remplacé par la compilation des dépôt distants, cependant l'utilisateur peut choisir de conserver les informations déjà présentes en les supposant à jour.


\subsection{Dépôt d'installation}
\label{subsection:depot_installation}
Une installation minecraft, définie par un dossier \textit{.minecraft}, est importée au besoin grâce aux fichiers \textit{mcmod.info} présents dans l'archive.
Elle se comporte virtuellement comme un véritable dépôt, ce qui permet de le comparer au dépôt local et de l'utiliser de manière transparente dans le système.



\subsection{Structure du dépôt}
\begin{figure}
\dirtree{%
.1 /.
.2 Mods.json.
.2 a.
.3 aether.
.4 aether.json.
.4 aether\_legacy-1.12.2-v1.4.4.
.3 appliedenergistics.
.4 appliedenergistics.json.
.4 Applied-Energistics-2-Mod-1.12.2.
.4 \dots
.2 c.
.3 cofhcore.
.4 cofhcore.json.
.2 t.
}
\caption{Exemple de hiérarchie d'un dépôt}
\label{fig:hierarchie}
\end{figure}

Mods.json (\ref{figure:mods.json}) contient la liste des modid proposés et les informations générales, indépendantes des versions.
Les champs obligatoires sont \textit{modid} et \textit{name}.

Pour un modid donnée, \textit{modid}.json (\ref{figure:modid.json}) contient la liste de toutes les versions connues et leurs informations particulières.
Notamment, pour chaque version si un lien de téléchargement est donné, il peut pointer vers le dépôt lui-même ou un site de téléchargement.
Les informations que l'on peut y trouver sont:
\begin{itemize}
    \item mcversion: \textbf{obligatoire}
    \item requiredMods: liste des mods et leurs versions requis pour le bon fonctionnement de celui-ci
    \item dependants: liste de mods qui pourrait utiliser celui-ci.
    \item urls: urls pour télécharger le fichier
    \item alias: peut servir à identifier les mods grâce à leur nom de fichier, dans le cas d'un échec de lecture de \textit{mcmod.info}.
\end{itemize}

\begin{figure}
\centering
\begin{verbatim}
[
    "<modid>": {
        "name": <name>,
        "description": <description>,
        "url": <url>,
        "updateJSON": <url>
    },
    "cofhcore": {
        "name": "Thermal Expansion"
    }
]
\end{verbatim}
\caption{Fichier Mods.json}
\label{figure:mods.json}
\end{figure}

\begin{figure}
\begin{verbatim}
{
    "<modversion>": {
        "mcversion": <mcversion>
        "urls": "<file url>",
        "requiredMods": [],
        "dependants": [],
        "alias": []
    }
}
\end{verbatim}
\caption{Fichier \textit{modid}.json}
\label{figure:modid.json}
\end{figure}


Tous les liens vers d'autres ressources seront des URLs.
À la différence d'une simple chaîne de caractères, les URLs fournissent toutes les informations sur la méthode à employer pour récupérer la ressource (http, ftp, \dots).
Les URLs spécifiés pour le téléchargement de fichier peuvent être seulement relatifs, alors la résolution de l'URL complet se fera par rapport à l'URL menant à la racine du dépôt (virtuel ou local).
Un URL relatif est même conseillé pour pouvoir déplacer facilement un dépot sur n'importe quel autre support, derrière n'importe quel protocol.
Tout URL partageant une partie commune avec la racine du dépôt sera réduit à l'URL relatif.
Ce système considère le dépôt local comme un potentiel dépôt distant pour d'autres utilisateurs.







\section{Dossier Minecraft}
L'instance de jeu utilisée dépend du lieu de lancement du programme ou des options passées en paramètres.
Par défaut, le programme remonte jusqu'au dossier \textit{.minecraft} le plus proche.
S'il n'est existe pas parmis les parents, tente d'utiliser le répertoire par défaut: \textit{.minecraft} sur Linux, ou \textit{AppData/.minecraft} sur windows.

Les informations sauvegardées localement comme décris au paragraphe \ref{section:depot} sont placées par défaut dans le répertoire \textit{.minecraft/forgemods}.
Ce répertoire sert de dépôt local et se présente sous la même forme qu'un dépôt en ligne pour simplifier l'importation des données.
Les informations du dépôt local sont toujours utilisées en premier, mais peuvent être mises à jour par les autres dépôts connus.

Lors de la synchronisations des informations avec un dépôt distant, aucun fichier de mod n'est téléchargé.
Si le dépôt fournis également les fichiers, leur disponibilité est conservée par l'URL et ils ne seront téléchargés que sur demande de l'utilisateur.


\subsection{État de l'installation}
Chaque version installée possède l'un des états d'installation suivant: AUTO si le mod est installé comme dépendance d'un autre, MANUEL si l'utilisateur l'a installé explicitement, VÉRROUILLÉ pour conserver le mod malgrès des erreurs.

Les installations manuelles sont l'ancrage qui maintient les fichiers en place.
Si un mod installé automatiquement n'est plus nécessaire, parce qu'aucun autre ne le requiert comme dépendance, il doit être retiré de l'installation.
Les mods vérrouillés ne doivent en aucun cas être modifiés sauf si l'utilisateur le décide (désinstallation).

\subsection{Cache}
Si l'utilisateur le permet, le dépôt local peut sauvegarder les fichiers originaux qui ont un jour été téléchargés ou détectés pour les mettre à disposition d'autres installations.
Par contre, lors de l'actualisation du dépôt local, les urls pointant vers le cache sont conservés.


\section{Commandes}
Paramètres génériques:
\begin{itemize}
    \item \texttt{----minecraft}: dossier d'installation minecraft (\$USER\_HOME/.minecraft)
    \item \texttt{----depot}: dossier de dépôt local (\$USER\_HOME/.minecraft/forgemods)
\end{itemize}

\medskip
Description de commandes utiles:
\begin{itemize}
    \item \texttt{show list [regex] [--v] [----all] [----installes]} affiche les mods connus: dans l'installation, et/ou dans le dépôt.
L'option --v permet d'afficher toutes les informations connues (ne pas combiner avec \texttt{----all}).
    \item \texttt{show dependencies [----absents]}: affiche la liste des dépendances pour les mods dans l'installation minecraft, ou seulement les absents.
    \item \texttt{show mod modid[@version]}: affiche les informations du dépôt relatives au mod.
    \item \texttt{install modid}: installe un nouveau mod grâce aux informations dans le dépôt local.
    \item \texttt{depot refresh}: importe et sauvegarde le dépôt pour vérifier son intégrité.
    \item \texttt{depot import [modid] [----all]}: importe les informations extraites des mods dans l'installation vers le dépôt local.
Necessite une installation minecraft.
    \item \texttt{depot udpate}: met à jour le dépôt en téléchargeant les informations sur les dépôts internet.
    \item \texttt{add--repository}: enregistre une nouvelle adresse internet de dépôt
\end{itemize}


\subsection{Installation}
Une installation cible correspond à l'union entre les mods déjà installés manuellement et les nouveaux mods demandés explicitement.
Ensuite il faut calculer l'arbre des dépendances et vérifier la disponibilité: les intervalles de version ne sont pas vide et contiennent une solution satisfaisante.
Si une demande n'est pas satisfaite pas une version déjà présente, le fichier correspondant est téléchargé et les précédantes versions sont supprimées.


\end{document}